% Options for packages loaded elsewhere
\PassOptionsToPackage{unicode}{hyperref}
\PassOptionsToPackage{hyphens}{url}
%
\documentclass[
  12pt,
  a4paper,
]{report}
\usepackage{amsmath,amssymb}
\usepackage{lmodern}
\usepackage{setspace}
\usepackage{iftex}
\ifPDFTeX
  \usepackage[T1]{fontenc}
  \usepackage[utf8]{inputenc}
  \usepackage{textcomp} % provide euro and other symbols
\else % if luatex or xetex
  \usepackage{unicode-math}
  \defaultfontfeatures{Scale=MatchLowercase}
  \defaultfontfeatures[\rmfamily]{Ligatures=TeX,Scale=1}
\fi
% Use upquote if available, for straight quotes in verbatim environments
\IfFileExists{upquote.sty}{\usepackage{upquote}}{}
\IfFileExists{microtype.sty}{% use microtype if available
  \usepackage[]{microtype}
  \UseMicrotypeSet[protrusion]{basicmath} % disable protrusion for tt fonts
}{}
\makeatletter
\@ifundefined{KOMAClassName}{% if non-KOMA class
  \IfFileExists{parskip.sty}{%
    \usepackage{parskip}
  }{% else
    \setlength{\parindent}{0pt}
    \setlength{\parskip}{6pt plus 2pt minus 1pt}}
}{% if KOMA class
  \KOMAoptions{parskip=half}}
\makeatother
\usepackage{xcolor}
\IfFileExists{xurl.sty}{\usepackage{xurl}}{} % add URL line breaks if available
\IfFileExists{bookmark.sty}{\usepackage{bookmark}}{\usepackage{hyperref}}
\hypersetup{
  hidelinks,
  pdfcreator={LaTeX via pandoc}}
\urlstyle{same} % disable monospaced font for URLs
\usepackage{longtable,booktabs,array}
\usepackage{calc} % for calculating minipage widths
% Correct order of tables after \paragraph or \subparagraph
\usepackage{etoolbox}
\makeatletter
\patchcmd\longtable{\par}{\if@noskipsec\mbox{}\fi\par}{}{}
\makeatother
% Allow footnotes in longtable head/foot
\IfFileExists{footnotehyper.sty}{\usepackage{footnotehyper}}{\usepackage{footnote}}
\makesavenoteenv{longtable}
\usepackage{graphicx}
\makeatletter
\def\maxwidth{\ifdim\Gin@nat@width>\linewidth\linewidth\else\Gin@nat@width\fi}
\def\maxheight{\ifdim\Gin@nat@height>\textheight\textheight\else\Gin@nat@height\fi}
\makeatother
% Scale images if necessary, so that they will not overflow the page
% margins by default, and it is still possible to overwrite the defaults
% using explicit options in \includegraphics[width, height, ...]{}
\setkeys{Gin}{width=\maxwidth,height=\maxheight,keepaspectratio}
% Set default figure placement to htbp
\makeatletter
\def\fps@figure{htbp}
\makeatother
\setlength{\emergencystretch}{3em} % prevent overfull lines
\providecommand{\tightlist}{%
  \setlength{\itemsep}{0pt}\setlength{\parskip}{0pt}}
\setcounter{secnumdepth}{5}
\usepackage[none]{hyphenat}
\pagestyle{plain}
\raggedbottom
\usepackage{hyperref}
\usepackage{floatpag}
\floatpagestyle{empty}
\usepackage{booktabs}
\usepackage{float}
\usepackage[document]{ragged2e} % left-justified text - comment for fully justified text
\usepackage{nonumonpart}
\usepackage[nottoc,numbib]{tocbibind}


\renewenvironment{abstract}{
\pagestyle{empty}

\begin{picture}(580,820)(80,-64)%
  
\put(58,  744){\makebox(100, 8)[l]{\abst@small Faculty of Sociasl Sciences}}
\put(289, 744){\makebox(100, 8)[l]{\abst@small Master's Programme in Economics}}
%\put(289, 744){\makebox(100, 8)[l]{\abst@small\@subject}}
\put(58,  714){\makebox(100, 8)[l]{\abst@small Theo Blauberg}}
\put(58,  684){\parbox[l]{450pt}{\renewcommand{\baselinestretch}{.9}\abst@small Tale of two discrepant shocks}}
\put(58,  654){\makebox(100, 8)[l]{\abst@small Master's Thesis}}
\put(212, 654){\makebox(100, 8)[l]{\abst@small July 2022}}
\put(366, 654){\makebox(100, 8)[l]{\abst@small 69}}
\put(58,  115) {\makebox(100, 8)[l]{\abst@small Jotain ja jotain}}
\put(58,  85) {\makebox(100, 8)[l]{\abst@small Kaisan kellari }}
\put(58,  59) {\makebox(100, 8)[l]{\abst@small Kuolin sisältä}}

\begin{@summary}\abst@small}
{\end{@summary}

\put(53,30){\framebox(462,746){}} % laatikko
\put(284,739){\line(0,1){37}} % pystyviiva
\put(53,739){\line(1,0){462}} 
\put(53,709){\line(1,0){462}}
\put(53,679){\line(1,0){462}}
\put(53,649){\line(1,0){462}}
\put(207,649){\line(0,1){30}} % pystyviiva
\put(361,649){\line(0,1){30}} % pystyviiva


\put(53,80){\line(1,0){462}}
\put(53,110){\line(1,0){462}}
\put(53,140){\line(1,0){462}}



\put(53,781){\makebox(100,8)[l]{\abst@small HELSINGIN YLIOPISTO --- HELSINGFORS UNIVERSITET --- UNIVERSITY OF HELSINKI}}
\put(58,767){\makebox(150,6)[l]{\tiny Tiedekunta --- Fakultet --- Faculty}}
\put(289,767){\makebox(100,6)[l]{\abst@tiny Koulutusohjelma --- Utbildningsprogram --- Degree programme}}
\put(58,730){\makebox(100,5)[l]{\abst@tiny Tekij\"a --- F\"orfattare --- Author}}
\put(58,700){\makebox(100,5)[l]{\abst@tiny Ty\"on nimi --- Arbetets titel --- Title}}
\put(58,670){\makebox(100,5)[l]{\abst@tiny Ty\"on laji --- Arbetets art --- Level}}
\put(212,670){\makebox(100,5)[l]{\abst@tiny Aika --- Datum --- Month and year }}
\put(366,670){\makebox(100,5)[l]{\abst@tiny Sivum\"a\"ar\"a --- Sidantal --- Number of pages}}
	
\put(58,640){\makebox(100,5)[l]{\abst@tiny Tiivistelm\"a --- Referat --- Abstract}}
\put(58,131){\makebox(100,5)[l]{\abst@tiny Avainsanat --- Nyckelord --- Keywords}}
\put(58,101){\makebox(100,5)[l]{\abst@tiny S\"ailytyspaikka --- F\"orvaringsst\"alle --- Where deposited}}
\put(58,71){\makebox(100,5)[l]{\abst@tiny Muita tietoja --- \"Ovriga uppgifter --- Additional information}}
\end{picture}

\clearpage
}
\ifLuaTeX
  \usepackage{selnolig}  % disable illegal ligatures
\fi
\usepackage[]{natbib}
\bibliographystyle{apalike}

\author{}
\date{\vspace{-2.5em}}

\begin{document}

\begin{titlepage}
    \begin{center}
        \vspace*{1cm}
        
        \includegraphics[width=0.3\textwidth]{latex/university.png}
        
        \vspace{1.5cm}
        
        \textbf{Tale of two discrepant shocks}
 
        \vspace{0.5cm}
         Response of the Finnish economy to the European union carbon policy shocks between 2005 and 2021 
             
        \vspace{0.5cm}
 
        \textbf{Theo Blauberg}
 
        \vfill
             
        A thesis presented for the degree of\\
        Master of Social Sciences
             
        \vspace{0.8cm}

    

    
        University of Helsinki\\
        Faculty of Social sciences\\
        Department of Economics\\
        June 2022

    \end{center}         
    
\end{titlepage}
{
\pagestyle{empty}

\begin{picture}(580,820)(80,-64)%

\put(58,  744){\makebox(100, 8)[l]{\abst@small Faculty of Social Sciences}}
\put(289, 744){\makebox(100, 8)[l]{\abst@small Master's Programme in Economics}}
%\put(289, 744){\makebox(100, 8)[l]{\abst@small\@subject}}
\put(58,  714){\makebox(100, 8)[l]{\abst@small Theo Blauberg}}
\put(58,  684){\parbox[l]{450pt}{\renewcommand{\baselinestretch}{.9}\abst@small Tale of two discrepant shocks}}
\put(58,  654){\makebox(100, 8)[l]{\abst@small Master's Thesis}}
\put(212, 654){\makebox(100, 8)[l]{\abst@small July 2022}}
\put(366, 654){\makebox(100, 8)[l]{\abst@small 69}}
\put(58,  115) {\makebox(100, 8)[l]{\abst@small Jotain ja jotain}}
\put(58,  85) {\makebox(100, 8)[l]{\abst@small Kaisan kellari }}
\put(58,  59) {\makebox(100, 8)[l]{\abst@small Kuolin sisältä}}

\begin{@summary}\abst@small}

    Summary of the main contents of the work: topic, methodology and results.
    
    Topics are classified according to the ACM Computing Classification System
    (CCS): check command \verb+\classification{}+. A small set of paths (1-3) should be used, starting from any top nodes
    referred to bu the root term CCS leading to the leaf nodes. The elements
    in the path are separated by right arrow, and emphasis of each element individually can be indicated
    by the use of bold face for high importance or italics for intermediate
    level. The combination of individual boldface terms may give the reader
    additional insight. 

{\end{@summary}

\put(53,30){\framebox(462,746){}} % laatikko
\put(284,739){\line(0,1){37}} % pystyviiva
\put(53,739){\line(1,0){462}} 
\put(53,709){\line(1,0){462}}
\put(53,679){\line(1,0){462}}
\put(53,649){\line(1,0){462}}
\put(207,649){\line(0,1){30}} % pystyviiva
\put(361,649){\line(0,1){30}} % pystyviiva


\put(53,80){\line(1,0){462}}
\put(53,110){\line(1,0){462}}
\put(53,140){\line(1,0){462}}



\put(53,781){\makebox(100,8)[l]{\abst@small HELSINGIN YLIOPISTO --- HELSINGFORS UNIVERSITET --- UNIVERSITY OF HELSINKI}}
\put(58,767){\makebox(150,6)[l]{\tiny Tiedekunta --- Fakultet --- Faculty}}
\put(289,767){\makebox(100,6)[l]{\abst@tiny Koulutusohjelma --- Utbildningsprogram --- Degree programme}}
\put(58,730){\makebox(100,5)[l]{\abst@tiny Tekij\"a --- F\"orfattare --- Author}}
\put(58,700){\makebox(100,5)[l]{\abst@tiny Ty\"on nimi --- Arbetets titel --- Title}}
\put(58,670){\makebox(100,5)[l]{\abst@tiny Ty\"on laji --- Arbetets art --- Level}}
\put(212,670){\makebox(100,5)[l]{\abst@tiny Aika --- Datum --- Month and year }}
\put(366,670){\makebox(100,5)[l]{\abst@tiny Sivum\"a\"ar\"a --- Sidantal --- Number of pages}}
	
\put(58,640){\makebox(100,5)[l]{\abst@tiny Tiivistelm\"a --- Referat --- Abstract}}
\put(58,131){\makebox(100,5)[l]{\abst@tiny Avainsanat --- Nyckelord --- Keywords}}
\put(58,101){\makebox(100,5)[l]{\abst@tiny S\"ailytyspaikka --- F\"orvaringsst\"alle --- Where deposited}}
\put(58,71){\makebox(100,5)[l]{\abst@tiny Muita tietoja --- \"Ovriga uppgifter --- Additional information}}
\end{picture}

\clearpage
}

{
\setcounter{tocdepth}{1}
\tableofcontents
}
\setstretch{1.5}
\hypertarget{intro}{%
\chapter{Introduction}\label{intro}}

The government reforms are a

\hypertarget{clipolfi}{%
\chapter{Climate policy in Finland}\label{clipolfi}}

In this chapter, I will provide a short introduction to the climate policy in Finland before the implementation of the European Union emission trading system and a brief explainer of the different phases of the European Union emission trading system. The effects of these regulatory updates are discussed in the following chapter, but I feel that short synopsis of the regulatory evolution is necessary for the reader to fully appreciate the economic consequences.

\hypertarget{preeuclipol}{%
\section{Prior to EU Emission trading system}\label{preeuclipol}}

Finland was the first country in the world to implement a tax on the sources of carbon emissions \citep{hallitus1989, bavbek2016}. The tax encompassed various emission sources, as it targeted both transport fuels and fuels used for energy production \citep{lin2011,ekins1999}. The carbon tax was stringent as it was indifferent to the final user and buyer, whether household or industrial, of the energy-producing fuels \citep{ekins1999}. Only fuel source exempted from this tax was wood \citep{elbaum2021}. The tax was initially modest with a relatively low valuation of 1.12 € per CO2 equivalent tonne and the tax was progressively increased to a more substantial level \citep{bavbek2016}. Sweden closely followed the Finnish example and enacted carbon tax in 1991 \citep{andersson2019}. The effectiveness of the Finnish and Swedish carbon taxes will be discussed in the next chapter where I will further discuss the literature and the evidence of the effectiveness of different policy choices.

Carbon tax was not implemented as a revenue raising measure, as it was from the beginning argued from environmental grounds as it could price the externalities cause by the burning of fossil fuels. The government proposal that was enacted as law in 1989 and entered into force in the beginning of 1990. \citep{hallitus1989}. The effects of these early carbon taxes are not in the scope of this thesis, but they might be the answer to my counterintuitive findings of the reaction of the Finnish economy to the carbon policy shocks of European Union. Finnish economy had more time and a solid monetary incentive to use the fossil fuel resources more effectively and to invest to green infrastructure prior to the enaction of the European Union emission trading system when comparing to other European economies which did not have such incentives. Even after that the tax continues to affect the sectors that are not subject to the European union emission trading system.

\hypertarget{eu-emission-trading-system}{%
\section{EU emission trading system}\label{eu-emission-trading-system}}

The second chapter of Finnish climate policy began in 2005 as the European Union emission trading system (ETS) was established. The ETS is based on a cap-and-trade scheme that restricts the total emissions for the entire affected sectors (these will vary in the different phases) and lets the market participants trade with each other \citep{garcia2021}. Also, various allocation schemes have been implemented in the different phases of the EU ETS as the allocation strategy has been updated \citep{verde2019free}. The ETS has been operating in four phases: Phase I ran from 2005 to 2007 and served as an experimentation period, Phase II operated from 2008 to 2012, Phase III from 2013 to 2020, and Phase IV will run from 2021 to 2030 \citep{ellerman2020,joltreau2019,verde2019free}.

Phase I of EU ETS was widely considered the experimentation period where the institutions of ETS were tested; also, in the first two phases, the national governments were left in charge to plan the allocation of these certificates to their respective industries \citep{verde2019free}. These national allocation plans (NAP) were then put under the scrutiny of the European Commission \citep{ellerman2020}. The verification procedure of the NAPs will be a significant source of the carbon policy surprise, which is discussed in chapter 4.

Phase II continued with a similar framework of NAPs and their Commission approvals \citep{ellerman2020}. The governments were allowed to auction up to 10 per cent of the allowances, compared to 5 per cent in Phase I \citep{ellerman2020}. Industrial production slowed abruptly after the financial crisis, which made the cap non-binding, thus reducing the price of the allowances to near zero \citep{verde2019free}. The effects of this can be seen clearly in Figure 1 below.

TÄHÄN PITÄÄÄ LISÄTÄ SE PLOTTI niiden hintojen kehityksestä ja sen aspektia pitää muuttaa

The oversupply of ETS allowances in late Phase II led to the reforms in Phase III. The most substantial updates to the ETS were the abolition of the NAP and the resulting centralisation of the system by adopting a single EU-wide cap. This cap was planned to reduce yearly by a linear amount that was decided to be 1.74 per cent of the year 2010 total allowances. \citep{ellerman2020}. This linear decrease would lead to a total of 21 per cent reduction by 2020 in emissions in the markets governed by the ETS when compared to the levels in 2005 \citep{verde2019free}. Another major reform enacted in Phase III was the phasing out of the free allocation to the energy sector in 2013 and plans of enacting this also to the remaining industrial sectors by 2027 \citep{ellerman2020}. The effects of these strict system overhauls can also be seen in figure 1, where the news of future updates can be seen moving the price of futures before it is realised at the spot price of the allowances. This is the essence behind the carbon policy surprise series and its usefulness in identifying the structural shocks in the SVAR model in chapter 5.

The changes brought by the Phase IV of the European Union ETS can be characterised by a more ambitious pace of allowance reductions and more stringent rules for the free allocations in the remaining sectors that still had them \citep{verde2019free}. This ambitious pace was countervailed by diverting the auction revenues to be used to support energy sector modernisation by the innovation fund and modernisation fund \citep{verde2019free}. As Phase IV has only recently begun, the full implications of the rule changes are yet to be seen. Another dimension that will test the resolve of the European decision-makers is higher energy costs and the effects of those on their constituents.

\hypertarget{rellit}{%
\chapter{Relevant literature}\label{rellit}}

In this chapter I will first present the previous empirical research on how environmental policy decisions have affected consumers, and economies more widely. The second subsection will cover a variety of previous research that utilised similar methodologies that I have used in this thesis. My aim in this chapter is to provide both a short background on the topic at hand and examples on previous of similar uses of similar methodological frameworks.

\hypertarget{effects}{%
\section{Effects of environmental policy}\label{effects}}

There has been some research on the effects of carbon taxes in different fields. Nevertheless, not one is done as extensively in Finland. Previously \citet{palanne2021} studied the effects of carbon taxes on Finnish passenger car traffic. They studied how tax schemes affected the carpool and if the break-in higher gasoline taxes would affect the types of cars people would buy \citep{palanne2021}. Unfortunately, their results were not encouraging; they estimated that between the years 2013 and 2017, the personal transport emissions decreased only about 2.3 percentages \citep{palanne2021}. Another paper written by \citet{sahari2019} studied how the Finnish consumers reacted to the changing electricity prices. She found out that the electricity prices had a noticeable impact on consumers heating choices who were building or renovating their houses \citep{sahari2019}. She used the variability of the electric prices across Finland to estimate the consumers' price elasticity between more environmentally conscious and more traditional household heating options \citep{sahari2019}.

Another research done in the context of Nordic countries that had contrasting results is an article written by \citet{andersson2019}. Author studied the Swedish economy's response to the enaction of carbon tax in the 1990s. Andersson used a quasi-experiment to find significant reductions in carbon emissions after implementing a carbon tax in Sweden in 1991 \citep{andersson2019}. The carbon tax started as relatively moderate fiscal policy but was later increased to more substantial heights \citep{andersson2019}. He also found that consumers reacted to carbon tax hikes more than just the market-driven price changes; these findings were achieved using a synthetic control \citep{andersson2019}. These findings are in interesting contention with the findings of Palanne and Sahari. Some studies have also concentrated on the Finnish economy in the same time period as Anderson was in Sweden. For example, a working paper written by \citet{elbaum2021} focuses on the response of the Finnish economy to carbon taxes in the 1990s. In uses a similar approach as Andersson to estimate the causal impact of the carbon tax \citep{elbaum2021}. He found a similar reaction in the Finnish economy as Andersson found in Sweden \citep{elbaum2021}. This could suggest that in the time of Palanne's and Sahari's research, the effects of carbon taxes were already embedded into the decision-making process of Finnish consumers. This also underlines the importance of understanding the historical processes behind the studies as the results seem to be substantially context-dependent.

Another study that takes a much broader view on the macroeconomic impacts of carbon taxes on European economies was written by \citet{metcalf2020}. They implemented a plethora of time-series analyses on different European countries prior and after they implemented different carbon taxes \citep{metcalf2020}. Interestingly they found negligible impacts on GDP growth or employment but a substantial reduction in greenhouse gas emissions \citep{metcalf2020}. \citet{metcalf2021} also published article where he outlines a theoretical framework on how carbon taxes and other greenhouse gas reduction mechanisms can work and reviews prior theoretical literature on the topic. He adds an essential point to the previous literature; if the effects of additional emissions are uncertain, the policymakers should lean more on cap-and-trade models as these have a hard limit on the number of emissions possible to emit and thus lowering the risk on possible tipping point scenarios \citep{metcalf2021}.

Similar miniscule effects to employment and total output as \citet{metcalf2020} were also reported by \citet{martin2014} when they studied the United Kingdom's carbon tax implementation on the industrial sector. They differentiated plants using micro-econometric methods that parsed out different tax burdens between them \citep{martin2014}. They did not find statistically significant evidence that this would have adverse outcomes to the treatment plants compared to their control counterparts \citep{martin2014}.

In their paper, \citet{acemoglu2016} reiterated the importance of a quick transition to decarbonise the economy. They based their model on microdata from the United States, which is a contrast to more macro-centric results of \citet{metcalf2020} and \citet{andersson2019}. The model formalised by \citeauthor{acemoglu2016} predicts that only using carbon taxes as the sole policy tool has a high welfare cost in the long run, especially when lowering the discount rate of the future welfare losses. They suggest that subsidies for clean technology and research were a cost-effective intervention to lower the disutility beared by the future generations \citep{acemoglu2016}. Their model had a much longer time horizon than other studies that can be found, running more than two hundred years, also they take a prescriptive approach as most of the studies described above are descriptive in nature \citep{acemoglu2016}. The optimal path they estimated relied heavily on the public investment in the less polluting technologies. They also stated that if they would relax the assumption of linear damages from the greenhouse gas pollution, their findings could also tilt to favour carbon taxes more \citep{acemoglu2016}. The research by Acemoglu also sidestep the possibility of the cap-and-trade schemes and the possible effects of these schemes. It could be argued that the cap-and-trade scheme is notionally equal to a carbon tax.

This is in stark contrast with the main inspiration of my thesis which is the research done by \citet{kaenzig2022}, who studied solely the effects of the European Union emission trading scheme on the European and, particularly, the United Kingdom's economy. He used the futures market as a tool to identify the structural shocks of carbon policy surprises in the constantly evolving carbon policy environment that is the European Union \citep{kaenzig2022}. He quantified the size of these carbon policy shocks to United Kingdom's economy using structural vector autoregressive model \citep{kaenzig2022}. This kind of surprise estimation has been previously used in the time series analysis of oil markets. As the futures are the markets' best guess for the future price of these carbon permits, given there are no transportation costs, and the risk tolerance of the seller and buyer are equal \citep{nakamura2018}. This way Känzig could estimate the surprise felt by the markets using tight time steps around policy announcements of European emission trading scheme \citep{kaenzig2022}. He found robust evidence on the carbon policy shock having substantial negative impacts on GDP growth and employment \citep{kaenzig2022}.

\hypertarget{prevsvar}{%
\section{Previous SVAR-IV research}\label{prevsvar}}

Prior to development of the instrument variable based identification of the structural vector autoregression method, which will be discussed at length in the next chapter. The structural vector autoregression models were widely used to analyse macroeconomic phenomena. They are an insightful tool to analyse dynamic interdependent systems. This is why they have been deployd for example in macroeconomics to study the responses to changes in monetary policy \citep{wolf2020svar}. Likewise, another widely researched field for macroeconomists is the reactions to the oil price shocks \citep{kilian2009not}.

Both previous research traditions that have utilised SVARs have in common that the identification of the shocks is difficult due to the endogenous nature of these shocks \citep{kaenzig2021}. An ingenious way of sidestepping this problem of endogeneity was answered by Känzig in his previous research in the realms of and oil and carbon policies \citep{kaenzig2022,kaenzig2021}. and this identification strategy will be further elaborated in the chapter x of instrument variable (Känzig, 2022).

The SVAR-IV was first introduced in lecture by \citet{stock2008s}. It was an ingenious way to identify the structural shocks by using an external instrument \citep{stock2008s}. The methodology was developed further in research by \citet{stock2012disentangling} where they illustrated the propagation channels of the recession of the years 2007-2009. The evidence presented in the research of Stock and Watson supports the idea that financial collapse and tight monetary policy had significant impact on the slow economic recovery \citep{stock2012disentangling}.

The SVAR-IV was used in the Finnish context by \citet{keranen2020identification}, where they examined the size of the fiscal multiplier of government spending. The most important finding in this study, according to the authors, was the formation of an instrument variable \citep{keranen2020identification}. This instrument variable can have use in future research. The authors indicated a note of caution whether the instrument should be used as an external instrument as used by \citet{stock2008s,stock2012disentangling} \citep{keranen2020identification}.

Prior to the research I am replicating in this thesis Känzig (2021) studied the effects of the news shocks by the Organisation of the Petroleum Exporting Countries (OPEC) by using similar methodology that he later used in the studying the effects EU Emission Trading system (Känzig, 2021 and Känzig 2022). To determine the effects of oil supply news shocks Känzig utilised the external instrument method as they are highly endogenous to wider macroeconomy (Känzig, 2021). This instrument was constructed by using a tight time frame around the OPEC oil supply news announcements to measure the movement of oil prices (Känzig, 2021). Using this strategy Känzig could build an exogenous and relevant instrument of the structural oil supply news shock (Känzig, 2021). I will elaborate the methodology further in the chapter x. Känzig identified both short- and long-term impulse responses to a negative oil supply news shock. In short-term the oil price increase was substantial, but with passage of time it decreases back to the original price level. In contrast, oil production and industrial production are not instantaneously affected, but in long run a significant decrease in all can be detected. (Känzig, 2021.)

De Winne and Peersman (2021), focused on their research in the effects of extreme weather events on the food prices and real GDP in both developed and developing countries. In their research they two different instruments to estimate the effects of extreme weather events to the global agricultural commodity market, and secondary impacts to economic variables. Their research argues that the middle-income countries would be hardest hit with price shocks in the agricultural commodity market, as lower-income countries which are more reliant on agriculture would incur windfall profits with higher prices. They also suggest that previous research has undervalued the consequences of extreme weather event related price shocks in the agricultural commodities markets can have in the rich countries. Even though the evidence is statistically the authors call for additional research to theorise the propagation channels to wider economy. (De Winne and Peersman, 2021.) The findings should be treated with caution as the authors analysis does not take account of the decrease in agricultural output capability that might coincide with the changing climate. Their research though is vital in underlining the effects to the high-income countries through the price shocks (De Winne and Peersman, 2021). With similar focus Faccia, Parker and Stracca (2021) studied in a European central bank working paper the effects of anomalous temperatures in winter and summer to medium term inflation. According to the authors, medium term inflation is non-trivially affected by the extreme weather events, in addition they also argue that climate change will have an affect to the central banks primary mandate of price stability. (Faccia et al., 2021.)

The original paper of using exogenous announcement shocks as time series instruments is the paper written by Romer and Romer in

Romerin ja Romerin paperi verojen muutoksien vaikutuksesta

Overall, these studies highlight the value of the SVAR-IV method as an instrument in trying to

sit se 2018 stock and watsoni ja mertens ravni

Sit se bayespaperi myös

\hypertarget{econometric-approach}{%
\chapter{Econometric approach}\label{econometric-approach}}

In this section I will describe the econometric model that I will use to identify the structural shocks using the Structural vector autoregression with instruments variables (SVAR-IV). In the formalisation of my model I will follow in the footseps of Känzig (2022) and Montiel Olea et al.~(2021).

\hypertarget{var}{%
\section{VAR}\label{var}}

Presume a standard VAR-model with a lag length of \(p\).

\begin{equation} 
y_t = b + B_1y_{t-1}+\dots+ B_py_{t-p}+ u_t
\label{eq:svar}
\end{equation}

Where the \(y_t\) refers to a \(n \times 1\) vector of the observed endogenous variables at time step \(t\). The \(B_1,\dots, B_p\) are \(n \times n\) coefficient matrices. \(u_t\) is an \(n \times 1\) vector of the reduced form innovations with a covariance matrix of \(\Sigma\).

\hypertarget{identification-of-the-structural-shocks}{%
\section{Identification of the structural shocks}\label{identification-of-the-structural-shocks}}

An integral assumption in using SVAR-models that, the one-step-ahead prediction errors i.e.~the innovations \(u_t\) are a linear combination of a vector of mutually orthogonal structural shocks \(\varepsilon_t\):

\[
u_t = S\varepsilon_t
\]

Where \(S\) is a nonsingular \(n \times n\) structural impact matrix. Due to the orthogonality the structural the \(n \times n\) covariance matrix of \(\text{var}(\varepsilon_t)=\Omega\) is diagonal. Thus due the linear mapping of the innovations and structural shocks described in the equation x, we can describe the covariance matrix of the innovations as:

\[
\Sigma = S \Omega S'
\]

For the sake of clarity, the \(\varepsilon_{1,t}\) is defined to describe the shock of interest, the carbon policy shock. Latter part of this chapter will present how by using a external instrument approach we can identify the structural impact vector \(s_1\) which is analogous to the first column of the structural impact matrix \(S\).

\hypertarget{external-instrument}{%
\section{External instrument}\label{external-instrument}}

For an external instrument \(z_t\) to be useful in idenfiying sturctural shocks it has to satisfy the following two conditions:

\[
\begin{aligned}
\mathbb{E}(z_t \varepsilon_{1,t}) &= \alpha \neq 0 \\
\mathbb{E}(z_t \varepsilon_{i\neq 1,t}) &= 0 
\end{aligned}
\]

The equation x is the relevance condition and the equation y is the exogeneity condition. If these conditions in tandem with the invertibility requirement are met the sign and the and scale of the \(s_1\) can be identified by:

\begin{equation}
s_1 \propto \frac{\mathbb{E}(z_t u_t)}{\mathbb{E}(z_t u_{1,t})}
\label{eq:prop}
\end{equation}

The size of \(\alpha\) is the strength of the external instrument and it can be tested with the XXX elaborated in the nönnönnöös (2018) article. After the structural impact vector has been identified the estimation of the confidence bands in IRF can be done with a moving block bootstrap method, also used in the Känzig (2022).

\hypertarget{comparing-other-identification-strategies}{%
\section{Comparing other identification strategies}\label{comparing-other-identification-strategies}}

Other possible strategies to identify the structural shocks would be to use heteroscedasticity based identification of structural vector autoregressions or local projections. In an interesting article Plagborg-Møller and Wolf (2021) offer a proof that local projections and SVARs are estimating the same impulse responses, but they have different finite-sample properties (Plagborg-Møller and, Wolf 2021).

When comparing the results of SVAR-IV to ones produced with local projection the variance of the impulse response functions are lower, but with a trade off of bias in the results if the VAR is noninvertible (Wolf, 2020). In the appendix XX I will provide the impulse responses that are produced via Local projection-instrument variable approach, as an robustness check for the results of the baseline SVAR-IV model. The results we see are at least notionally similar and thus provide additional evidence that the baseline model can be trusted.

Even though these both would have been a valid choice as an instrument, but as one the main tasks of this thesis is to quasi-replicate the findings of Känzig I will continue with the SVAR-Iv that my findings are as comparable as possible. Additional reasons why I selected the SVAR-IV as my approach was for the reliability and the efficiency, which are paramount in estimating the responses to a shock from a short sample.

\hypertarget{data}{%
\chapter{Data}\label{data}}

I will follow the Känzig's formulation as having the following endogenous variables. The model consists of two different sections, the carbon section, which consists of consumer price index of energy and the disaggregated greenhouse gas emission time series. The macroeconomic section which is consists of head line consumer price index, industrial production index, 3 month Euribor rate, unemployment rate, OMX Helsinki stock index, and real exchange rate of Finland.

\[
y_t =
\begin{bmatrix}
\text{Energy consumer price index} \\
\text{GHG emissions} \\
\text{Consumer price index} \\
\text{Industrial production} \\
\text{3 month Euribor} \\
\text{Unemployment rate} \\
\text{OMX Helsinki} \\
\text{Real exchange rate}
\end{bmatrix}
\]

The sources of these endogenous variables can be found from the appendix X. The sample dates of my variables are from the begining of year 2000 to the end of third quarter in 2021. All the endogenous variables are also reported in monthly timeseries. The Greenhouse gas emissions are an exception to this and they have to be disaggregated to a monthly time series.

Following Känzig (2022) example all the variables have been, except the 3 month Euribor and Unemployment rate stored as log-levels. The reasoning behind all of these choices is elaborated in the following subchapters.

\hypertarget{greenhouse-gas-emission-disaggregation}{%
\section{Greenhouse gas emission disaggregation}\label{greenhouse-gas-emission-disaggregation}}

The greenhouse gas emission data is reported annually, due to the commitments that were made by the United Nations Framework Convention on Climate Change (Stat Fin, 2022). This produces a problem that has to be adressed as our model will be build on a monthly time series.

Känzig solved the problem by using the Chow-Lin dissaggregation method. Accuracy of the disaggregation can be increased by additional relevant indicators that are reported in the desired frequency and are also correlated with the target values (Chow and Lin, 1971). Känzig used as his indicators the Consumer price index of energy products and industrial production index. As can be seen from the Figure 1, the Finnish non-renewable energy production is highly seasonal.

\begin{figure}
\centering
\includegraphics{Slide_pictures/energy.png}
\caption{Monthly energy production in Finland by energy source}
\end{figure}

As a first impulse I wanted to capture this seasonal variation to my disaggregated time series. That is why I produced three different dissaggregated time series of greenhouse gas emissions: a dummy disaggregation without indicators, with similar indicators that Känzig used, and with the additional information of the amounts of non-renewable energy production. The results of these three disaggregations can be seen from Figure 2.

\begin{figure}
\centering
\includegraphics{Slide_pictures/ghg_plot.png}
\caption{Different disaggregation strategies from yearly values of the GHG emissions in Finland to monthly}
\end{figure}

The dummy disaggregation strategy produces a yearly value divided by 12 as it's estimation, this can be considered also as the reported value, when we analyse the other two estimates. The disaggregation produced following in the footsteps of Känzig produces a relatively smooth time series that could be understood as a trend time series. The final disaggregation is the one with additional information. The values are varying wildly between summer months and winter months, this is due to the variation in the usage of non-renewable energy sources, that can be seen in the figure 1.

Even though the my estimate might be more truthful in capturing the actual monthly greenhouse gas emissions. It also produces more noise to the model and in the next subchapter I will discuss the problems of not using trended values. In appendix x can be seen the impulse response functions that are produced with my estimate, and how it produces seasonal noise.

\hypertarget{using-trend-values}{%
\section{Using trend values}\label{using-trend-values}}

Känzig did not utilise trend values in his analysis, this might not been a significant problem, as he used values that were observed from Europe. This means that the seasonal variation was much lower than in the data that was observed from Finland. The seasonal variation of employment can be seen in the figure 3:

\begin{figure}
\centering
\includegraphics{Slide_pictures/unemp_comparison.png}
\caption{Monthly observed unemployment rate and the seasonally adjusted trend value in Finland}
\end{figure}

Using the original observed unemployment rate bring similar problems as using the disaggregation with additional information. It brings noise to the impulse responses. The in the appendix x can be seen the impulse responses when using the observed values. In the impulse response is present a substantial contemporaneous shock to the job market that is highly unlikely and thus the model might have captured seasonal variation to the impulse responses of the shocks. These similar problems are also present with the industrial production index.

\begin{figure}
\centering
\includegraphics{Slide_pictures/prod_comparison.png}
\caption{Monthly observed industrial production index and the seasonally adjusted trend values in Finland}
\end{figure}

The problems are similar whether the variable is industrial production index, unemployment rate, or the disaggregated greenhouse gas emissions. The variation that is due to eihter measurement errors, seasonal variation, or the inherent randomness that is not produced by the processes we want to detect. Especially when trying to infer long and medium-term effects of the carbon policy shock, the short-term variation of the endogenous variables affect the accuracy greatly. This can be seen when comparing the impulse response functions of the actual model in the chapter x and the one produced with the observed data in the appendix x.

\hypertarget{using-log-levels}{%
\section{Using log-levels}\label{using-log-levels}}

MIKSIIIIII?

\begin{itemize}
\item
  ne antaa korvaamatonta tietoa mitä ei voida saada vain muutoksen suuruudesta.
\item
  tämä tietysti antaa
\end{itemize}

\newpage

\hypertarget{instrument}{%
\chapter{Instrument}\label{instrument}}

Juttuja instrumentin käyttämisestä asian mittaamiseks jota ei voi mitata yleistä teoriaa instrumentti muuttujista

\hypertarget{futures-market}{%
\section{Futures market}\label{futures-market}}

The usage of futures markets relies on the hypothesis that the markets will effectively incorporate all available information to the price discovery mechanism (Hayek, 1945). That would then

Standard theory on asset price formation which was formulated by Pindyck (2001) predicts that futures contracts on a day \(d\) with a maturity \(h\) are valued as:

\[
  F_d^h = \mathbb{E}_d(P_{d+h})-RP^h_d
\]

Are a combination of the expected price of the asset \(P_{d+h}\) with the information available on the date \(d\) and the risk premium. It is apparent from the equation

\hypertarget{high-frequency-identification}{%
\section{High-frequency identification}\label{high-frequency-identification}}

I will follow the Känzig's formulation with the high-frequency identification of the instrument. It is based on similar approach that was used by Romer and Romer (2010), and Känzig (2021) in his previous research on the consequences of the OPEC announcements. The surprise series of the carbon futures market is calculated from the log differences in the daily closing price of the EU emission trading certificate futures.

It can be assumed that because of the tight identification period the changes in the risk premium are not changing \(RP_d \approx RP_{d-1}\) and thus the surprise series is representing the updates in the expected future price of the emission trading certificates.

\[
\begin{split}
  \text{Surprise}_d &= F^h_d-F^h_{d-1} \\
  &= \mathbb{E}_d(P_{d+h})-RP^h_d-\mathbb{E}_{d-1}(P_{d+h})+RP^h_{d-1} \\
  &= \mathbb{E}_d(P_{d+h})-\mathbb{E}_{d-1}(P_{d+h}) 
\end{split}
\]

This daily surprise series is the aggregated to a monthly carbon policy surprise series. A indicator function \(1_{cp}(d)\) whether a day contains an carbon policy regulatory event is used to mask days with now regulatory events to zero. The regulatory events are listed in Appendix B.

\[
\text{CPSurprise}_m = \sum_{d\in m}\text{Surprise}_d 1_{cp}(d)
\]

This monthly surprise series is an ideal external instrument due to the exogeneity resulting from the short time frame of the identification. This means that a

\hypertarget{regulatory-dates}{%
\section{Regulatory Dates}\label{regulatory-dates}}

The identification of regulatory events is a vital part of the building of this instrument. In the figure 1 the values of the \(\text{CPSurprise}_m\) can be clearly seen.

\begin{figure}
\centering
\includegraphics{Slide_pictures/surprise.png}
\caption{Figure 3: Carbon policy surprise series}
\end{figure}

The dates are collected prior to ???

\hypertarget{diagnostics-of-the-surprise-as-an-external-instrument}{%
\section{Diagnostics of the Surprise as an external instrument}\label{diagnostics-of-the-surprise-as-an-external-instrument}}

The major problem that the instrument could have is that it would have a is the serial correlation. THankfully there is no evidence of persistent autocorrelation. There is a statistically significant autocorrelation after 11 lags. It can be seen from the following figure.

\begin{figure}
\centering
\includegraphics{Slide_pictures/acf.png}
\caption{Autocorrelation of the instrument}
\end{figure}

This autocorrelation of 11 lags would mean that the last year's regulatory announcements have are correlating with the announcements at time step \(t\). This can be seen from the bar plot below. As the announcements are not equally distributed between months our results may capture some seasonal impacts if the seasonal variation is not adequately handled.

\begin{figure}
\centering
\includegraphics{Slide_pictures/shock_amounts.png}
\caption{Monthly annoucements and their effects to the future prices}
\end{figure}

This is an additional reason why there is noticeable seasonal fluctuation in the appendix x.

\newpage

\hypertarget{results}{%
\chapter{Results}\label{results}}

\hypertarget{first-stage}{%
\section{First stage}\label{first-stage}}

Inference with instrumental variables lies on the assumptions of relevancy with the shock of interest and exogeneity with other shocks. An other hidden assumption is that a weak correlation between the instrument and the structural shock of interest compromises the large sample validity of the inference based on the instrument. Due to this Montiel Olea, Stock and Watson (2020) propose that the first-stage heteroskedasticity-robust F-statistic between the instrument and the VAR-residual should be reported as a measure of the strength of the instrument. They also propose a rule of thumb of \(F > 10\) for to be sure that the instrument is not weak.(Montiel Olea, Stock and Watson, 2020.)

For my model the \(F = 10.99\), which would imply that the instrument is strong enough to be used with standard inference and there is no need to use the weak-instrument robust confidence sets that were elaborated further in Montiel Olea, Stock and Watson (2020). With the first stage regression we also find that the instrument explains \(2.97\%\) of the energy price index residual. In conclusion there is no evidence to consider that a weak instrument problem would affect the inference.

\hypertarget{effects-to-finnish-macroeconomy}{%
\section{Effects to Finnish macroeconomy}\label{effects-to-finnish-macroeconomy}}

In the following section I will present the resulting impulse response functions to a carbon policy shock. The solid black line is a point estimate and the darker and lighter regions are the 68 and 90 per cent confidence bands respectfully. These confidence bands are calculated with a moving block bootstrap which was first brought forth by Jentsch and Lunsdorf (2019). \footnote{The major advantage of moving block bootstrap comparing it to wild bootstrap is that it will produce accurate confidence intervals. As the wild bootstrap will produce unaccurate impulse response functions to SVAR-IV (Jentsch and Lunsdorf, 2016). The code used in my thesis is heavily relying on Känzig (2021) reproduction files found \href{https://github.com/dkaenzig/replicationOilSupplyNews}{here}. If found all the mistakes are naturally mine.} With a block size of 20 and with \(10 000\) bootstrap replications.

\begin{figure}
\centering
\includegraphics{Slide_pictures/final_irf.png}
\caption{Impulse responses to a Carbon policy shock}
\end{figure}

The negative carbon policy shock is normalised to have a effect of \(1\%\) in increasing the price of energy. As the energy components, greenhouse gas emission, headline HICP, industrial output index, OMX Helsinki 25 stock index and the real broad exchange rate index are all handled in log-levels can the plots be interpreted as percentual changes. In contrast the unemployment rate and the 3 month Euribor interest rate are handled in percentage points and thus the plots represent changes of percentage points.

What can be clearly seen in this figure is the high persistency of the higher prices as a response to the carbon policy shock. This persistency in higher energy prices is feeding into higher headline consumer price index and it is statistically significant until the end of the estimation period, \(50\) months ahead. The behaviour of the consumer price indexes are inline with Känzig's (2022) findings from European wide context. Where my findings differ, and seem to defy the conventional wisdom, is the behaviour of the Finnish economy in the short-run.

There is an rather unexpected, but statistically insignificant, evidence that the Finnish economy benefits from the carbon policy shocks in the short-run. This can be seen as the industrial output index which is slightly elevated after the shock. But then continues to decrease to a statistically significant long term negative impulse. Similar, but opposite signed, impulse is with the unemployment rate. Starting with a statistically insignificant negative simultaneous impact that peaks around 10 months and then steady rise to a statistically significant positive reaction. The point estimate would imply that a carbon policy shock that would instantaneously raise the energy prices by 1\% would in \(40\) months time lead to a \(0.07\) percentage point rise in Finnish unemployment.

There is also statistically significant evidence that the carbon policy shock affects the 3 months Euribor interest rate, by reducing it with a point estimate of 0.1 percentage point in the long run. The effects of the carbon policy shock to the stock market seem to be statistically inconclusive. There is a noticeable evidence of effects to the Finnish real effective exchange rate in the long run.

\hypertarget{discussion-of-the-findings}{%
\section{Discussion of the findings}\label{discussion-of-the-findings}}

The reasons behind these impulses are outside the bounds of this thesis. But for the sake of future research I will provide few hypotheses. I would suggest, that there are two different reactions at play. A short run reaction that stimulates the economy and a long run reaction that is slowing down the economy. One hypothetical reason behind the short run stimulation might arise from the long history of Finnish carbon taxes which as Elbaum (2021) states made the Finnish industry more energy efficient. This energy efficiency would give Finnish industry a competetive advantage to its peers. Another possible explanation is the composition of Finnish industry \footnote{According to \href{https://www.stat.fi/til/tti/2020/tti_2020_2021-07-01_tie_001_en.html}{Statistics Finland} 45\% of the value produced by Finnish industry is from metal industry products.} with the high share of capital goods in the production basket. The short term stimulus emerge due to the other European countries investing to more sustainable capital which would, in turn make the products of Finnish industrial companies more attractive. These explanations are hardly exclusive and thus their relative share of effect or the existence of additional explanations will be left to future research.

The long term effects of these shocks are more in line with the findings of Känzig (2022). This could be hypothesised to mean that the short term stimulus of some kind is running out and Finland will revert to the European norm. This would imply that the lower aggregate demand in Europe, which is demonstrated by Känzig, would after the end of investment surge bring the Finnish economy to a lower long term level. Also as with the short term effects the long term will have to be left to the future research as it is well beyond the scope of my thesis.

\hypertarget{conclusion}{%
\chapter{Conclusion}\label{conclusion}}

  \bibliography{bibliography.bib}
\addcontentsline{toc}{chapter}{\bibname}

\end{document}
