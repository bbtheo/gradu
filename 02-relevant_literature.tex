% Options for packages loaded elsewhere
\PassOptionsToPackage{unicode}{hyperref}
\PassOptionsToPackage{hyphens}{url}
%
\documentclass[
]{article}
\usepackage{amsmath,amssymb}
\usepackage{lmodern}
\usepackage{iftex}
\ifPDFTeX
  \usepackage[T1]{fontenc}
  \usepackage[utf8]{inputenc}
  \usepackage{textcomp} % provide euro and other symbols
\else % if luatex or xetex
  \usepackage{unicode-math}
  \defaultfontfeatures{Scale=MatchLowercase}
  \defaultfontfeatures[\rmfamily]{Ligatures=TeX,Scale=1}
\fi
% Use upquote if available, for straight quotes in verbatim environments
\IfFileExists{upquote.sty}{\usepackage{upquote}}{}
\IfFileExists{microtype.sty}{% use microtype if available
  \usepackage[]{microtype}
  \UseMicrotypeSet[protrusion]{basicmath} % disable protrusion for tt fonts
}{}
\makeatletter
\@ifundefined{KOMAClassName}{% if non-KOMA class
  \IfFileExists{parskip.sty}{%
    \usepackage{parskip}
  }{% else
    \setlength{\parindent}{0pt}
    \setlength{\parskip}{6pt plus 2pt minus 1pt}}
}{% if KOMA class
  \KOMAoptions{parskip=half}}
\makeatother
\usepackage{xcolor}
\IfFileExists{xurl.sty}{\usepackage{xurl}}{} % add URL line breaks if available
\IfFileExists{bookmark.sty}{\usepackage{bookmark}}{\usepackage{hyperref}}
\hypersetup{
  hidelinks,
  pdfcreator={LaTeX via pandoc}}
\urlstyle{same} % disable monospaced font for URLs
\usepackage[margin=1in]{geometry}
\usepackage{longtable,booktabs,array}
\usepackage{calc} % for calculating minipage widths
% Correct order of tables after \paragraph or \subparagraph
\usepackage{etoolbox}
\makeatletter
\patchcmd\longtable{\par}{\if@noskipsec\mbox{}\fi\par}{}{}
\makeatother
% Allow footnotes in longtable head/foot
\IfFileExists{footnotehyper.sty}{\usepackage{footnotehyper}}{\usepackage{footnote}}
\makesavenoteenv{longtable}
\usepackage{graphicx}
\makeatletter
\def\maxwidth{\ifdim\Gin@nat@width>\linewidth\linewidth\else\Gin@nat@width\fi}
\def\maxheight{\ifdim\Gin@nat@height>\textheight\textheight\else\Gin@nat@height\fi}
\makeatother
% Scale images if necessary, so that they will not overflow the page
% margins by default, and it is still possible to overwrite the defaults
% using explicit options in \includegraphics[width, height, ...]{}
\setkeys{Gin}{width=\maxwidth,height=\maxheight,keepaspectratio}
% Set default figure placement to htbp
\makeatletter
\def\fps@figure{htbp}
\makeatother
\setlength{\emergencystretch}{3em} % prevent overfull lines
\providecommand{\tightlist}{%
  \setlength{\itemsep}{0pt}\setlength{\parskip}{0pt}}
\setcounter{secnumdepth}{5}
\usepackage[none]{hyphenat}
\pagestyle{plain}
\raggedbottom
\usepackage{hyperref}
\usepackage{floatpag}
\floatpagestyle{empty}
\usepackage{booktabs}
\usepackage{float}
\usepackage[document]{ragged2e} % left-justified text - comment for fully justified text
\usepackage{nonumonpart}
\usepackage[nottoc,numbib]{tocbibind}


\renewenvironment{abstract}{
\pagestyle{empty}

\begin{picture}(580,820)(80,-64)%
  
\put(58,  744){\makebox(100, 8)[l]{\abst@small Faculty of Sociasl Sciences}}
\put(289, 744){\makebox(100, 8)[l]{\abst@small Master's Programme in Economics}}
%\put(289, 744){\makebox(100, 8)[l]{\abst@small\@subject}}
\put(58,  714){\makebox(100, 8)[l]{\abst@small Theo Blauberg}}
\put(58,  684){\parbox[l]{450pt}{\renewcommand{\baselinestretch}{.9}\abst@small Tale of two discrepant shocks}}
\put(58,  654){\makebox(100, 8)[l]{\abst@small Master's Thesis}}
\put(212, 654){\makebox(100, 8)[l]{\abst@small July 2022}}
\put(366, 654){\makebox(100, 8)[l]{\abst@small 69}}
\put(58,  115) {\makebox(100, 8)[l]{\abst@small Jotain ja jotain}}
\put(58,  85) {\makebox(100, 8)[l]{\abst@small Kaisan kellari }}
\put(58,  59) {\makebox(100, 8)[l]{\abst@small Kuolin sisältä}}

\begin{@summary}\abst@small}
{\end{@summary}

\put(53,30){\framebox(462,746){}} % laatikko
\put(284,739){\line(0,1){37}} % pystyviiva
\put(53,739){\line(1,0){462}} 
\put(53,709){\line(1,0){462}}
\put(53,679){\line(1,0){462}}
\put(53,649){\line(1,0){462}}
\put(207,649){\line(0,1){30}} % pystyviiva
\put(361,649){\line(0,1){30}} % pystyviiva


\put(53,80){\line(1,0){462}}
\put(53,110){\line(1,0){462}}
\put(53,140){\line(1,0){462}}



\put(53,781){\makebox(100,8)[l]{\abst@small HELSINGIN YLIOPISTO --- HELSINGFORS UNIVERSITET --- UNIVERSITY OF HELSINKI}}
\put(58,767){\makebox(150,6)[l]{\tiny Tiedekunta --- Fakultet --- Faculty}}
\put(289,767){\makebox(100,6)[l]{\abst@tiny Koulutusohjelma --- Utbildningsprogram --- Degree programme}}
\put(58,730){\makebox(100,5)[l]{\abst@tiny Tekij\"a --- F\"orfattare --- Author}}
\put(58,700){\makebox(100,5)[l]{\abst@tiny Ty\"on nimi --- Arbetets titel --- Title}}
\put(58,670){\makebox(100,5)[l]{\abst@tiny Ty\"on laji --- Arbetets art --- Level}}
\put(212,670){\makebox(100,5)[l]{\abst@tiny Aika --- Datum --- Month and year }}
\put(366,670){\makebox(100,5)[l]{\abst@tiny Sivum\"a\"ar\"a --- Sidantal --- Number of pages}}
	
\put(58,640){\makebox(100,5)[l]{\abst@tiny Tiivistelm\"a --- Referat --- Abstract}}
\put(58,131){\makebox(100,5)[l]{\abst@tiny Avainsanat --- Nyckelord --- Keywords}}
\put(58,101){\makebox(100,5)[l]{\abst@tiny S\"ailytyspaikka --- F\"orvaringsst\"alle --- Where deposited}}
\put(58,71){\makebox(100,5)[l]{\abst@tiny Muita tietoja --- \"Ovriga uppgifter --- Additional information}}
\end{picture}

\clearpage
}
\ifLuaTeX
  \usepackage{selnolig}  % disable illegal ligatures
\fi
\usepackage[]{natbib}
\bibliographystyle{plainnat}

\author{}
\date{\vspace{-2.5em}}

\begin{document}

\begin{titlepage}
    \begin{center}
        \vspace*{1cm}
        
        \includegraphics[width=0.3\textwidth]{latex/university.png}
        
        \vspace{1.5cm}
        
        \textbf{Tale of two discrepant shocks}
 
        \vspace{0.5cm}
         Response of the Finnish economy to the European union carbon policy shocks between 2005 and 2021 
             
        \vspace{0.5cm}
 
        \textbf{Theo Blauberg}
 
        \vfill
             
        A thesis presented for the degree of\\
        Master of Social Sciences
             
        \vspace{0.8cm}

    

    
        University of Helsinki\\
        Faculty of Social sciences\\
        Department of Economics\\
        June 2022

    \end{center}         
    
\end{titlepage}
{
\pagestyle{empty}

\begin{picture}(580,820)(80,-64)%

\put(58,  744){\makebox(100, 8)[l]{\abst@small Faculty of Social Sciences}}
\put(289, 744){\makebox(100, 8)[l]{\abst@small Master's Programme in Economics}}
%\put(289, 744){\makebox(100, 8)[l]{\abst@small\@subject}}
\put(58,  714){\makebox(100, 8)[l]{\abst@small Theo Blauberg}}
\put(58,  684){\parbox[l]{450pt}{\renewcommand{\baselinestretch}{.9}\abst@small Tale of two discrepant shocks}}
\put(58,  654){\makebox(100, 8)[l]{\abst@small Master's Thesis}}
\put(212, 654){\makebox(100, 8)[l]{\abst@small July 2022}}
\put(366, 654){\makebox(100, 8)[l]{\abst@small 69}}
\put(58,  115) {\makebox(100, 8)[l]{\abst@small Jotain ja jotain}}
\put(58,  85) {\makebox(100, 8)[l]{\abst@small Kaisan kellari }}
\put(58,  59) {\makebox(100, 8)[l]{\abst@small Kuolin sisältä}}

\begin{@summary}\abst@small}

    Summary of the main contents of the work: topic, methodology and results.
    
    Topics are classified according to the ACM Computing Classification System
    (CCS): check command \verb+\classification{}+. A small set of paths (1-3) should be used, starting from any top nodes
    referred to bu the root term CCS leading to the leaf nodes. The elements
    in the path are separated by right arrow, and emphasis of each element individually can be indicated
    by the use of bold face for high importance or italics for intermediate
    level. The combination of individual boldface terms may give the reader
    additional insight. 

{\end{@summary}

\put(53,30){\framebox(462,746){}} % laatikko
\put(284,739){\line(0,1){37}} % pystyviiva
\put(53,739){\line(1,0){462}} 
\put(53,709){\line(1,0){462}}
\put(53,679){\line(1,0){462}}
\put(53,649){\line(1,0){462}}
\put(207,649){\line(0,1){30}} % pystyviiva
\put(361,649){\line(0,1){30}} % pystyviiva


\put(53,80){\line(1,0){462}}
\put(53,110){\line(1,0){462}}
\put(53,140){\line(1,0){462}}



\put(53,781){\makebox(100,8)[l]{\abst@small HELSINGIN YLIOPISTO --- HELSINGFORS UNIVERSITET --- UNIVERSITY OF HELSINKI}}
\put(58,767){\makebox(150,6)[l]{\tiny Tiedekunta --- Fakultet --- Faculty}}
\put(289,767){\makebox(100,6)[l]{\abst@tiny Koulutusohjelma --- Utbildningsprogram --- Degree programme}}
\put(58,730){\makebox(100,5)[l]{\abst@tiny Tekij\"a --- F\"orfattare --- Author}}
\put(58,700){\makebox(100,5)[l]{\abst@tiny Ty\"on nimi --- Arbetets titel --- Title}}
\put(58,670){\makebox(100,5)[l]{\abst@tiny Ty\"on laji --- Arbetets art --- Level}}
\put(212,670){\makebox(100,5)[l]{\abst@tiny Aika --- Datum --- Month and year }}
\put(366,670){\makebox(100,5)[l]{\abst@tiny Sivum\"a\"ar\"a --- Sidantal --- Number of pages}}
	
\put(58,640){\makebox(100,5)[l]{\abst@tiny Tiivistelm\"a --- Referat --- Abstract}}
\put(58,131){\makebox(100,5)[l]{\abst@tiny Avainsanat --- Nyckelord --- Keywords}}
\put(58,101){\makebox(100,5)[l]{\abst@tiny S\"ailytyspaikka --- F\"orvaringsst\"alle --- Where deposited}}
\put(58,71){\makebox(100,5)[l]{\abst@tiny Muita tietoja --- \"Ovriga uppgifter --- Additional information}}
\end{picture}

\clearpage
}

{
\setcounter{tocdepth}{2}
\tableofcontents
}
\hypertarget{rellit}{%
\section{Relevant literature}\label{rellit}}

In this chapter I will first present the previous empirical research on how environmental policy decisions have affected consumers, and economies more widely. The second subsection will cover a variety of previous research that utilised similar methodologies that I have used in this thesis. My aim in this chapter is to provide both a short background on the topic at hand and examples on previous of similar uses of similar methodological frameworks.

\hypertarget{effects}{%
\subsection{Effects of environmental policy}\label{effects}}

There has been some research on the effects of carbon taxes in different fields. Nevertheless, not one is done as extensively in Finland. Previously \citet{palanne2021} studied the effects of carbon taxes on Finnish passenger car traffic. They studied how tax schemes affected the carpool and if the break-in higher gasoline taxes would affect the types of cars people would buy \citep{palanne2021}. Unfortunately, their results were not encouraging; they estimated that between the years 2013 and 2017, the personal transport emissions decreased only about 2.3 percentages \citep{palanne2021}. Another paper written by \citet{sahari2019} studied how the Finnish consumers reacted to the changing electricity prices. She found out that the electricity prices had a noticeable impact on consumers heating choices who were building or renovating their houses \citep{sahari2019}. She used the variability of the electric prices across Finland to estimate the consumers' price elasticity between more environmentally conscious and more traditional household heating options \citep{sahari2019}.

Another research done in the context of Nordic countries that had contrasting results is an article written by \citet{andersson2019}. Author studied the Swedish economy's response to the enaction of carbon tax in the 1990s. Andersson used a quasi-experiment to find significant reductions in carbon emissions after implementing a carbon tax in Sweden in 1991 \citep{andersson2019}. The carbon tax started as relatively moderate fiscal policy but was later increased to more substantial heights \citep{andersson2019}. He also found that consumers reacted to carbon tax hikes more than just the market-driven price changes; these findings were achieved using a synthetic control \citep{andersson2019}. These findings are in interesting contention with the findings of Palanne and Sahari. Some studies have also concentrated on the Finnish economy in the same time period as Anderson was in Sweden. For example, a working paper written by \citet{elbaum2021} focuses on the response of the Finnish economy to carbon taxes in the 1990s. In uses a similar approach as Andersson to estimate the causal impact of the carbon tax \citep{elbaum2021}. He found a similar reaction in the Finnish economy as Andersson found in Sweden \citep{elbaum2021}. This could suggest that in the time of Palanne's and Sahari's research, the effects of carbon taxes were already embedded into the decision-making process of Finnish consumers. This also underlines the importance of understanding the historical processes behind the studies as the results seem to be substantially context-dependent.

Another study that takes a much broader view on the macroeconomic impacts of carbon taxes on European economies was written by \citet{metcalf2020}. They implemented a plethora of time-series analyses on different European countries prior and after they implemented different carbon taxes \citep{metcalf2020}. Interestingly they found negligible impacts on GDP growth or employment but a substantial reduction in greenhouse gas emissions \citep{metcalf2020}. \citet{metcalf2021} also published article where he outlines a theoretical framework on how carbon taxes and other greenhouse gas reduction mechanisms can work and reviews prior theoretical literature on the topic. He adds an essential point to the previous literature; if the effects of additional emissions are uncertain, the policymakers should lean more on cap-and-trade models as these have a hard limit on the number of emissions possible to emit and thus lowering the risk on possible tipping point scenarios \citep{metcalf2021}.

Similar non-significant effects to employment and total output as \citet{metcalf2020} were also reported by \citet{martin2014} when they studied the United Kingdom's carbon tax implementation on the industrial sector. They differentiated plants using micro-econometric methods that parsed out different tax burdens between them \citep{martin2014}. They did not find statistically significant evidence that this would have adverse outcomes to the treatment plants compared to their control counterparts \citep{martin2014}.

In their paper, \citet{acemoglu2016} reiterated the importance of a quick transition to decarbonise the economy. They based their model on microdata from the United States, which is a contrast to more macro-centric results of \citet{metcalf2020} and \citet{andersson2019}. The model formalised by \citeauthor{acemoglu2016} predicts that only using carbon taxes as the sole policy tool has a high welfare cost in the long run, especially when lowering the discount rate of the future welfare losses. They suggest that subsidies for clean technology and research were a cost-effective intervention to lower the disutility beared by the future generations \citep{acemoglu2016}. Their model had a much longer time horizon than other studies that can be found, running more than two hundred years, also they take a prescriptive approach as most of the studies described above are descriptive in nature \citep{acemoglu2016}. The optimal path they estimated relied heavily on the public investment in the less polluting technologies. They also stated that if they would relax the assumption of linear damages from the greenhouse gas pollution, their findings could also tilt to favour carbon taxes more \citep{acemoglu2016}. The research by Acemoglu also sidestep the possibility of the cap-and-trade schemes and the possible effects of these schemes. It could be argued that the cap-and-trade scheme is notionally equal to a carbon tax.

This is in stark contrast with the main inspiration of my thesis which is the research done by \citet{kaenzig2022}, who studied solely the effects of the European Union emission trading scheme on the European and, particularly, the United Kingdom's economy. He used the futures market as a tool to identify the structural shocks of carbon policy surprises in the constantly evolving carbon policy environment that is the European Union \citep{kaenzig2022}. He quantified the size of these carbon policy shocks to United Kingdom's economy using structural vector autoregressive model \citep{kaenzig2022}. This kind of surprise estimation has been previously used in the time series analysis of oil markets. As the futures are the markets' best guess for the future price of these carbon permits, given there are no transportation costs, and the risk tolerance of the seller and buyer are equal \citep{nakamura2018}. This way Känzig could estimate the surprise felt by the markets using tight time steps around policy announcements of European emission trading scheme \citep{kaenzig2022}. He found robust evidence on the carbon policy shock having substantial negative impacts on GDP growth and employment \citep{kaenzig2022}.

\hypertarget{prevsvar}{%
\subsection{Previous SVAR-IV research}\label{prevsvar}}

Prior to development of the instrument variable based identification of the structural vector autoregression method, which will be discussed at length in the next chapter. The structural vector autoregression models were widely used to analyse macroeconomic phenomena. They are an insightful tool to analyse dynamic interdependent systems. This is why they have been deployd for example in macroeconomics to study the responses to changes in monetary policy \citep{wolf2020svar}. Likewise, another widely researched field for macroeconomists is the reactions to the oil price shocks \citep{kilian2009not}.

Both previous research traditions that have utilised SVARs have in common that the identification of the shocks is difficult due to the endogenous nature of these shocks \citep{kaenzig2021}. An ingenious way of sidestepping this problem of endogeneity was answered by Känzig in his previous research in the realms of and oil and carbon policies \citep{kaenzig2022,kaenzig2021}. and this identification strategy will be further elaborated in the chapter x of instrument variable (Känzig, 2022).

The SVAR-IV was first introduced in lecture by \citet{stock2008s}. It was an ingenious way to identify the structural shocks by using an external instrument \citep{stock2008s}. The methodology was developed further in research by \citet{stock2012disentangling} where they illustrated the propagation channels of the recession of the years 2007-2009. The evidence presented in the research of Stock and Watson supports the idea that financial collapse and tight monetary policy had significant impact on the slow economic recovery \citep{stock2012disentangling}.

The SVAR-IV was used in the Finnish context by \citet{keranen2020identification}, where they examined the size of the fiscal multiplier of government spending. The most important finding in this study, according to the authors, was the formation of an instrument variable \citep{keranen2020identification}. This instrument variable can have use in future research. The authors indicated a note of caution whether the instrument should be used as an external instrument as used by \citet{stock2008s,stock2012disentangling} \citep{keranen2020identification}.

Prior to the research I am replicating in this thesis Känzig (2021) studied the effects of the news shocks by the Organisation of the Petroleum Exporting Countries (OPEC) by using similar methodology that he later used in the studying the effects EU Emission Trading system (Känzig, 2021 and Känzig 2022). To determine the effects of oil supply news shocks Känzig utilised the external instrument method as they are highly endogenous to wider macroeconomy (Känzig, 2021). This instrument was constructed by using a tight time frame around the OPEC oil supply news announcements to measure the movement of oil prices (Känzig, 2021). Using this strategy Känzig could build an exogenous and relevant instrument of the structural oil supply news shock (Känzig, 2021). I will elaborate the methodology further in the chapter x. Känzig identified both short- and long-term impulse responses to a negative oil supply news shock. In short-term the oil price increase was substantial, but with passage of time it decreases back to the original price level. In contrast, oil production and industrial production are not instantaneously affected, but in long run a significant decrease in all can be detected. (Känzig, 2021.)

De Winne and Peersman (2021), focused on their research in the effects of extreme weather events on the food prices and real GDP in both developed and developing countries. In their research they two different instruments to estimate the effects of extreme weather events to the global agricultural commodity market, and secondary impacts to economic variables. Their research argues that the middle-income countries would be hardest hit with price shocks in the agricultural commodity market, as lower-income countries which are more reliant on agriculture would incur windfall profits with higher prices. They also suggest that previous research has undervalued the consequences of extreme weather event related price shocks in the agricultural commodities markets can have in the rich countries. Even though the evidence is statistically the authors call for additional research to theorise the propagation channels to wider economy. (De Winne and Peersman, 2021.) The findings should be treated with caution as the authors analysis does not take account of the decrease in agricultural output capability that might coincide with the changing climate. Their research though is vital in underlining the effects to the high-income countries through the price shocks (De Winne and Peersman, 2021). With similar focus Faccia, Parker and Stracca (2021) studied in a European central bank working paper the effects of anomalous temperatures in winter and summer to medium term inflation. According to the authors, medium term inflation is non-trivially affected by the extreme weather events, in addition they also argue that climate change will have an affect to the central banks primary mandate of price stability. (Faccia et al., 2021.)

The original paper of using exogenous announcement shocks as time series instruments is the paper written by Romer and Romer in

Romerin ja Romerin paperi verojen muutoksien vaikutuksesta

Overall, these studies highlight the value of the SVAR-IV method as an instrument in trying to

sit se 2018 stock and watsoni ja mertens ravni

Sit se bayespaperi myös

\end{document}
