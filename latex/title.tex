\begin{titlepage}
    \begin{center}
        \vspace*{1cm}
        
        \includegraphics[width=0.3\textwidth]{latex/university.png}
        
        \vspace{1.5cm}
        
        \textbf{Tale of two discrepant shocks}
 
        \vspace{0.5cm}
         Response of the Finnish economy to the European union carbon policy shocks between 2005 and 2021 
             
        \vspace{0.5cm}
 
        \textbf{Theo Blauberg}
 
        \vfill
             
        A thesis presented for the degree of\\
        Master of Social Sciences
             
        \vspace{0.8cm}

    

    
        University of Helsinki\\
        Faculty of Social sciences\\
        Department of Economics\\
        June 2022

    \end{center}         
    
\end{titlepage}
{
\pagestyle{empty}

\begin{picture}(580,820)(80,-64)%

\put(58,  744){\makebox(100, 8)[l]{\abst@small Faculty of Social Sciences}}
\put(289, 744){\makebox(100, 8)[l]{\abst@small Master's Programme in Economics}}
%\put(289, 744){\makebox(100, 8)[l]{\abst@small\@subject}}
\put(58,  714){\makebox(100, 8)[l]{\abst@small Theo Blauberg}}
\put(58,  684){\parbox[l]{450pt}{\renewcommand{\baselinestretch}{.9}\abst@small Tale of two discrepant shocks}}
\put(58,  654){\makebox(100, 8)[l]{\abst@small Master's Thesis}}
\put(212, 654){\makebox(100, 8)[l]{\abst@small July 2022}}
\put(366, 654){\makebox(100, 8)[l]{\abst@small 69}}
\put(58,  115) {\makebox(100, 8)[l]{\abst@small Jotain ja jotain}}
\put(58,  85) {\makebox(100, 8)[l]{\abst@small Kaisan kellari }}
\put(58,  59) {\makebox(100, 8)[l]{\abst@small Kuolin sisältä}}

\begin{@summary}\abst@small}

    Summary of the main contents of the work: topic, methodology and results.
    
    Topics are classified according to the ACM Computing Classification System
    (CCS): check command \verb+\classification{}+. A small set of paths (1-3) should be used, starting from any top nodes
    referred to bu the root term CCS leading to the leaf nodes. The elements
    in the path are separated by right arrow, and emphasis of each element individually can be indicated
    by the use of bold face for high importance or italics for intermediate
    level. The combination of individual boldface terms may give the reader
    additional insight. 

{\end{@summary}

\put(53,30){\framebox(462,746){}} % laatikko
\put(284,739){\line(0,1){37}} % pystyviiva
\put(53,739){\line(1,0){462}} 
\put(53,709){\line(1,0){462}}
\put(53,679){\line(1,0){462}}
\put(53,649){\line(1,0){462}}
\put(207,649){\line(0,1){30}} % pystyviiva
\put(361,649){\line(0,1){30}} % pystyviiva


\put(53,80){\line(1,0){462}}
\put(53,110){\line(1,0){462}}
\put(53,140){\line(1,0){462}}



\put(53,781){\makebox(100,8)[l]{\abst@small HELSINGIN YLIOPISTO --- HELSINGFORS UNIVERSITET --- UNIVERSITY OF HELSINKI}}
\put(58,767){\makebox(150,6)[l]{\tiny Tiedekunta --- Fakultet --- Faculty}}
\put(289,767){\makebox(100,6)[l]{\abst@tiny Koulutusohjelma --- Utbildningsprogram --- Degree programme}}
\put(58,730){\makebox(100,5)[l]{\abst@tiny Tekij\"a --- F\"orfattare --- Author}}
\put(58,700){\makebox(100,5)[l]{\abst@tiny Ty\"on nimi --- Arbetets titel --- Title}}
\put(58,670){\makebox(100,5)[l]{\abst@tiny Ty\"on laji --- Arbetets art --- Level}}
\put(212,670){\makebox(100,5)[l]{\abst@tiny Aika --- Datum --- Month and year }}
\put(366,670){\makebox(100,5)[l]{\abst@tiny Sivum\"a\"ar\"a --- Sidantal --- Number of pages}}
	
\put(58,640){\makebox(100,5)[l]{\abst@tiny Tiivistelm\"a --- Referat --- Abstract}}
\put(58,131){\makebox(100,5)[l]{\abst@tiny Avainsanat --- Nyckelord --- Keywords}}
\put(58,101){\makebox(100,5)[l]{\abst@tiny S\"ailytyspaikka --- F\"orvaringsst\"alle --- Where deposited}}
\put(58,71){\makebox(100,5)[l]{\abst@tiny Muita tietoja --- \"Ovriga uppgifter --- Additional information}}
\end{picture}

\clearpage
}